Grid-connected distributed energy resource (DER) deployment has been in the rise in recent years. This has led to the introduction of highly variable distributed generation (DG) and energy storage system (ES) into the distribution grid. Currently, the DERs connected to the grid are deployed under a net-metering program or power purchase agreements. With the introduction of advanced metering infrastructure (AMI), more information regarding the system status are also available. There has also been significant advances in DG and load forecasting \cite{LOAD_FOR1,LOAD_FOR2,LOAD_FOR3,WInd_for_1,ospina2019forecasting}. Due to these recent advances, there is significantly more data available regarding the present and future status of the distribution grid. An important subject of research is how to make an effective use of all these information to make an optimal decision. There is a need for a solution which can dynamically optimize the available DERs based on present and forecasted system status while maintaining system constraints. Due to the intermittent nature of variable DGs and system constraints, this is a difficult problem to solve. The most common approaches found in the literature are discussed here. Some researchers use neural network based approaches to solve the issue \cite{OFF_LINE_1}. Researchers in \cite{off_2,off_3,off_4} try to tackle the issue with off-line day ahead planning. Authors in \cite{ANEW1,ANEW2,ANEW3} provided solutions based on genetic algorithm and particle swarm optimization that takes in pregenerated day ahead predictions of load and DG generation. All these methods either rely vastly on pregenerated profiles or only use real-time system status to optimize the available DERs. 

Another challenge in developing a state of the art algorithm to tackle this issue is the validation process. A pilot implementation of the control algorithm under test is considered the most reliable validation technique. In this case, sections of the actual system is used to validate the control strategy under test. Results of these validations are highly realistic but require much time and resource investment to implement. Also, the risk of physical harm that can come from directly implementing novel control approaches in a real electrical infrastructure makes this approach less preferable \cite{TB_1}. An alternative or a reasonable previous step before executing the pilot test is a simulation-based validation where the power system is simulated in real-time, and the controller is interfaced with the simulated system through realistic protocols. This provides a less costly alternative for acceptable validation of algorithm before permanent deployment or pilot testing \cite{TB_1}. Researchers in \cite{TB_2} have implemented a test bed with TCP/IP communication between the real-time simulation and control equipment. The National Renewable Energy Laboratory (NREL) and the Pacific Northwest National Laboratory (PNNL) have hardware and software-based re-configurable test beds designed to handle various testing in this area \cite{NREL_DER_TEST, VOLTRON}. The researchers in \cite{TB_3, TB_5} have also developed cyber-physical testbeds for algorithm validation under cyber attacks. A detailed review on the available testbed developments can be found in \cite{TB_ALL}. The testbeds mentioned above are specific to actual settings and require resources and time for reconfiguration. 

From the discussion thus far, it is evident that there is a need for a real-time ESM solution that can optimize the long-term operating costs of a system containing DG and an energy storage (ES) system. There is also the need for developing proper validation platform for validating the behavior of such an algorithm before field deployment.  This paper presents an optimum real-time control strategy for such systems. The proposed control strategy takes into account the present and forecasted states of the system, together with the real-time price (RTP). The optimization is formulated as a graph search problem by discretizing the state of charge of the energy storage. Formulating the problem as a graph search enables the use of A* search algorithm to find the minimum cost of the graph. A* search algorithm \cite{a8book} is a popular search algorithm used in robotics and computer science usually to find the shortest path through a graph. This discretized graph search approach is capable of much faster computation compared to traditional optimization methods. This makes it suitable for real-time energy management.
Moreover, controller hardware-in-the-loop (CHIL) validation has been performed to validate the algorithm. The real-time simulation model used in the CHIL uses smart meters to measure system status and provides the data to the controller through the IEEE 1815 DNP3 communication protocol. The testbed uses an API (application programming interface) to connect to the DNP3 communication layer. This makes the operation of the testbed very easy to be reconfigured for generic use. The rest of the paper is structured as follows. Section II discusses the graph search based energy management solution. Section III  presents the details about the CHIL testbed and the implementation of DNP3 based communication. Section IV presents the results, and Section V presents the conclusions.


